%!TEX root=../../main.tex
\begin{chapterpage}{Inference for categorical data}
  \chaptertitle{Inference for\\[4mm]categorical data}
  \label{inferenceForCategoricalData}
  \chaptersection{singleProportion}
  \chaptersection{differenceOfTwoProportions}
  \chaptersection{twoWayTablesAndChiSquare}
  \chaptersection{oneWayChiSquare}
  \chaptersection{caseControStudies}
  \chaptersection{infForPropNotes}
  \chaptersection{infForPropExercises}
\end{chapterpage}
\renewcommand{\chapterfolder}{ch_inference_for_props_oi_biostat}

\chapterintro{{\large Previous chapters discussed methods of inference for numerical data; in this chapter, those methods are extended to categorical data, such as binomial proportions or data in two-way tables. While various details of the methods may change, such as the calculations for a test statistic or the distributions used to find a $p$-value, the core ideas and principles behind inference remain the same.  

Categorical data arise frequently in medical research because disease outcomes and patient characteristics are often recorded in natural categories such as types of treatment received, whether or not disease advanced to a later stage, or whether or not a patient responded initially to a treatment. In the simplest settings, a binary outcome (yes/no, success/failure, etc)  is recorded for a single group of participants, in hopes of learning more about the population from which the participants were drawn.  The binomial distribution is often used for the statistical model in this setting, and inference about the binomial probability of success provides information about a population proportion $p$.   In more complex settings, participant characteristics are recorded in a categorical variable with two or more levels, and the outcome or response variable itself has two or more levels.  In these instances, data are usually summarized in two-way tables with two or more rows and two or more columns.

As  with all methods of inference, it is important to understand how the data were collected and whether the data may be viewed as a random sample from a well-identified population, at least approximately.  This issue is at least as important as the formulas for test statistics and confidence intervals, and is often overlooked.

Be careful about the notation in this chapter---since $p$ is the standard notation for a population proportion and for a probability, $p$ does double duty in this chapter as a population parameter and significance level.}}


%__________________
\section{Inference for a single proportion}
\label{singleProportion}

Advanced melanoma is an aggressive form of skin cancer that until recently was almost uniformly fatal.  In rare instances, a patient's melanoma stopped progressing or disappeared altogether when the patient's immune system successfully mounted a response to the cancer. Those observations led to research into therapies that might trigger an immune response in cancer.  Some of the most notable successes have been in melanoma, particularly with two new therapies, nivolumab and ipilimumab\footnote{The -mab suffix in these therapies stands for monoclonal antibody, a therapeutic agent made by identical immune cells that are all clones of a unique parent cell from a patient.}.

A 2013 report in the New England Journal of Medicine by Wolchok et al. reported the results of a study in which patients were treated with both nivolumab and ipilimumab.\footnote{N Engl J Med 2013;369:122-33. DOI: 10.1056/NEJMoa1302369}   Fifty-three patients were given the new regimens concurrently, and the response to therapy could be evaluated in 52 of the 53.  Of the 52 evaluable patients, 21 (40\%) experienced a response according to commonly accepted criteria.  In previous studies, the proportion of patients responding to one of these agents was 30\% or less.  How might one compare the new data to past results?

The data from this study are binomial data, with success defined as a response to therapy. Suppose the number of patients who respond in a study like this is represented by the random variable $X$, where $X$ is binomial with parameters $n$ (the number of trials, where each trial is represented by a patient) and $p$ (the unknown population proportion of response). From formulas discussed in Chapter~\ref{modeling}, the mean of $X$ is $np$ and the standard deviation of $X$ is $\sqrt{np(1-p)}$.

Inference about $p$ is based on the sample proportion $\hat{p}$, where $\hat{p} = X/n$. In this case, $\hat{p} = 21/52 = 0.404$. If the sample proportion is nearly normally distributed, the normal approximation to the binomial distribution can be used to conduct inference; this method is commonly used.  When $X$ does not have an approximately normal distribution, exact inference can  based on the binomial distribution for $X$.  Both the normal approximation and exact methods are covered in this chapter.

\subsection{Inference using the normal approximation}
\label{normalApproxSingleBinomial}

A \term{sample proportion} can be described as a sample mean. If each success in the melanoma data  is represented as a \texttt{1} and each failure as a \texttt{0}, then the sample proportion is the mean of the 52 numerical outcomes:
\begin{eqnarray*}
\hat{p} = \frac{\ 0 + 1 + 1 + \cdots + 0\ }{52} = 0.404.
\end{eqnarray*}
The distribution of $\hat{p}$ is nearly normal when the distribution of successes and failures is not too strongly skewed.

\textD{\newpage}

\index{sampling distribution!sample proportion}

\begin{onebox}{Conditions for the sampling distribution of $\pmb{\hat{\MakeLowercase{p}}}$ being nearly normal}
The sampling distribution for $\hat{p}$, calculated from a sample of size $n$ from a population with a success proportion $p$, is nearly normal when
\begin{enumerate}
\item the sample observations are independent and
\item at least 10 successes and 10 failures are expected in the sample, i.e. $np\geq10$ and $n(1-p)\geq10$. This is called the \term{success-failure condition}.
\end{enumerate}
If these conditions are met, then the sampling distribution of $\hat{p}$ is approximately normal with mean $p$ and standard error
\index{standard error (SE)!single proportion}
\begin{eqnarray}
SE_{\hat{p}} = \sqrt{\frac{p(1-p)}{n}}.
\label{seOfPHat}
\end{eqnarray}
\end{onebox}%
\marginpar[\raggedright\vspace{-53mm}

$\hat{p}$\vspace{0mm}\\\footnotesize sample\\proportion\vspace{3mm}\\\normalsize$p$\vspace{0mm}\\\footnotesize population\\proportion]{\raggedright\vspace{-53mm}

$\hat{p}$\vspace{0mm}\\\footnotesize sample\\proportion\vspace{3mm}\\\normalsize$p$\vspace{0mm}\\\footnotesize population\\proportion}

When conducting inference, the population proportion $p$ is unknown. Thus, to construct a confidence interval, the sample proportion $\hat{p}$ can be substituted for $p$ to check the success-failure condition and compute the standard error. In a hypothesis test, $p_0$ is substituted for $p$.

\subsubsection{Confidence intervals for a proportion}
\label{confIntForPropSection}

\index{point estimate!single proportion}
\index{confidence interval!single proportion}

When using the normal approximation to the sampling distribution of $\hat{p}$, a confidence interval for a proportion has the same structure as a confidence interval for a mean; it is centered at the point estimate, with a margin of error calculated from the standard error and appropriate $z^{\star}$ value.  The formula for a 95\% confidence interval is
\[
  \hat{p} \pm 1.96 \sqrt{\frac{\hat{p}(1-\hat{p})}{n}}.
\]

\begin{examplewrap}
\begin{nexample}{Using the normal approximation, construct an approximate 95\% confidence interval for the response probability for patients with advanced melanoma who were administered the combination of nivolumab and ipilimumab.}

The independence and success-failure assumptions should be checked first.  Since the outcome of one patient is unlikely to influence that of other patients, the observations are independent.  The success-failure condition is satisfied since $n\hat{p} = (52)(.404) = 21  > 10$ and $n\hat{p}(1 - \hat{p}) = (52)(.596) = 31  > 10$.

The point estimate for the response probability, based on a sample of size $n = 52$, is $\hat{p} = 0.404$. For a 95\% confidence interval, $z^{\star} = 1.96$. The standard error is estimated as: $\sqrt{\frac{\ \hat{p}(1-\hat{p})\ }{n}} = \sqrt{\frac{(0.404)(1-0.404)}{52}} = 0.068$.  The confidence interval is
\[0.404 \pm 1.96 (0.068) \rightarrow (0.27, 0.54) \]
The approximate 95\% confidence interval for $p$, the population response probability of melanoma patients to the combination of these new drugs, is (0.27, 0.54) or (27\%, 54\%).  
\end{nexample}
\end{examplewrap}

\textD{\newpage}

\begin{exercisewrap}
\begin{nexercise}
In New York City on October 23rd, 2014, a doctor who had recently been treating Ebola patients in Guinea went to the hospital with a slight fever and was subsequently diagnosed with Ebola. Soon after, a survey conducted by the Marist Poll, an organization with a carefully designed methodology for drawing random samples from identified populations, found that 82\% of New Yorkers favored a "mandatory 21-day quarantine for anyone who has come in contact with an Ebola patient."\footnotemark{} a) Verify that the sampling distribution of $\hat{p}$ is nearly normal. b) Construct a 95\% confidence interval for $p$, the proportion of New York adults who supported a quarantine for anyone who has come into contact with an Ebola patient.\footnotemark{}
\end{nexercise}
\end{exercisewrap}
\addtocounter{footnote}{-1}%
\footnotetext{\oiRedirect{textbook-maristpoll_ebola_201410}{Poll ID NY141026 on maristpoll.marist.edu}.}%
\addtocounter{footnote}{1}%
\footnotetext{a) The poll is based on a simple random sample and consists of fewer than 10\% of the adult population of New York, which makes independence a reasonable assumption. The success-failure condition is satisfied since, $1042(0.82) > 5$ and $1042(1-0.82) > 5$. b) $0.82 \pm 1.96\sqrt{\frac{0.82(1-0.82)}{1042}} \rightarrow (0.796, 0.844)$.}

Did the participants in the melanoma trial constitute a random sample?  Patients who participate in clinical trials are unlikely to be a random sample of patients with the disease under study since the patients or their physicians must be aware of the trial, and patients must be well enough to travel to a major medical center and be willing to receive an experimental therapy that may have serious side effects.  

Investigators in the melanoma trial were aware that the observed proportion of patients responding in a clinical trial may be different than the hypothetical response probability in the population of patients with advanced melanoma. Study teams try to minimize these systematic differences by following strict specifications for deciding whether patients are eligible for a study. However, there is no guarantee that the results observed in a sample will be replicated in the general population.

Small, initial studies in which there is no control group, like the one described here, are early steps in exploring the value of a new therapy and are used to justify further study of a treatment when the results are substantially different than expected.  The largest observed response rate in previous trials of 30\% was close to the lower bound of the confidence interval from the study (27\%, 54\%), so the results were considered adequate justification for continued research on this treatment.

\subsubsection{Hypothesis testing for a proportion}
\label{htForPropSection}
\index{hypothesis testing!single proportion}

Just as with inference for population means, confidence intervals for population proportions can be used when deciding whether to reject a null hypothesis. It is useful in most settings, however, to calculate the $p$-value for a test as a measure of the strength of the evidence contradicting the null hypothesis.

When using the normal approximation for the distribution of $\hat{p}$ to conduct a hypothesis test, one should always verify that $\hat{p}$ is nearly normal under $H_0$ by checking the independence and success-failure conditions. Since a hypothesis test is based on the distribution of the test statistic under the null hypothesis, the success-failure condition is checked using the null proportion $p_0$, not the estimate $\hat{p}$. 

According to the normal approximation to the binomial distribution, the number of successes in $n$ trials is normally distributed with mean $np_0$ and standard deviation $\sqrt{np(1-p_0)}$. This approximation is valid when $np_0$ and $n(1-p_0)$ are both at least 10.\footnote{The normal approximation to the binomial distribution was discussed in Section~\ref{binomialModel} of Chapter~\ref{modeling}.} 

\textD{\newpage}

Under the null hypothesis, the sample proportion $\hat{p} = X/n$ is approximately distributed as 
\[N \left(p_0, \sqrt{\frac{p_0(1-p_0)}{n}} \right).\]
The test statistic $z$ for the null hypothesis $H_0: p = p_0$ based on a sample of size $n$ is 
\begin{align*}
  z &= \dfrac{\text{point estimate - null value}}{SE} \\
    &= \dfrac{\hat{p} - p_0}{\sqrt{\frac{(p_0)(1-p_0)}{n}}}. 
\end{align*}

\begin{examplewrap}
\begin{nexample}{Suppose that out of a cohort of 120 patients with stage 1 lung cancer at the Dana-Farber Cancer Institute (DFCI) treated with a new surgical approach, 80 of the patients survive at least 5 years, and suppose that National Cancer Institute statistics indicate that the 5-year survival probability for stage 1 lung cancer patients nationally is 0.60. Do the data collected from 120 patients support the claim that the DFCI population treated with this new form of surgery has a different 5-year survival probability than the national population? Let $\alpha = 0.10$, since this is an early study of the new surgery.}

Test the hypothesis $H_0: p = 0.60$ versus the alternative, $H_A:  p \neq 0.60$, using $\alpha = 0.10$. If we assume that the outcome of one patient at DFCI does not influence the outcome of other patients, the independence condition is met, and the success-failure condition is satisfied since $(120)(0.60) = 80 > 5$ and $(120)(1-0.60) = 40 > 5.$ The test statistic is the $z$-score of the point estimate: 
\[z = \dfrac{\text{point estimate - null value}}{SE} = \dfrac{0.67 - 0.60}{\sqrt{\frac{(0.60)(1-0.60)}{120}}} = 1.57. \]
The $p$-value is the probability that a standard normal variable is larger than 1.57 or smaller than -1.57, $P(|Z| > 1.57) = 0.12$ ; since the $p$-value is greater than 0.10, there is insufficient evidence to reject $H_0$ in favor of $H_A$. There is not convincing evidence that the survival probability at DFCI differs from the national survival probability.  Had a more traditional 0.05 significance level been used, the data would be even less convincing. 
\end{nexample}
\end{examplewrap}

\begin{examplewrap}
\begin{nexample}{Using the data from the study in advanced melanoma, use the normal approximation to the sampling distribution of $\hat{p}$ to test the null hypothesis that the response probability to the novel combined therapy is 30\% against a one-sided alternative that the response proportion is greater than 30\%. Let $\alpha = 0.10$.}

The test statistic has value 
\[
z = (0.404 - 0.30)/\sqrt{(0.30)(0.70)/52} = 1.64. 
\] 
The one-sided $p$-value is $P(Z \geq 1.64) = 0.05$; there is sufficient evidence to reject the null hypothesis at $\alpha = 0.10$. This is an example of where a two-sided test and a one-sided test yield different conclusions.
\end{nexample}
\end{examplewrap}

\index{data!nhanes}

\textD{\newpage}

\begin{exercisewrap}
\begin{nexercise}
One of the questions on the National Health and Nutrition Examination Survey (introduced in Chapter~\ref{inferenceForNumericalData}) asked participants whether they participated in moderate or vigorous intensity sports, fitness, or recreational activities. In a random sample of 135 adults, 76 answered "Yes" to the question. Based on this evidence, are a majority of American adults physically active?\footnotemark{}
\end{nexercise}
\end{exercisewrap}
\footnotetext{The observations are independent. Check success-failure: $np_0 = n(1-p_0) = 135(0.5) > 10$. $H_0: p = 0.5$; $H_A: p > 0.5$. Calculate the $z$-score: $z = \frac{0.56 - 0.50}{\sqrt{\frac{0.5(1-0.5)}{135}}} = 1.39$. The $p$-value is 0.08. Since the $p$-value is larger than 0.05, there is insufficient evidence to reject $H_0$; there is not convincing evidence that a majority of Americans are physically active, although the data suggest that may be the case.}


\subsection{Inference using exact methods}
\label{exactMethodsSingleBinomial}

When the normal approximation to the distribution of $\hat{p}$ may not be accurate, inference is based on exact binomial probabilities. Calculating confidence intervals and $p$-values based on the binomial distribution can be done by hand, with tables of the binomial distribution, or (more easily and accurately) with statistical software. The logic behind computing a $p$-value is discussed here, but the formulas for a confidence interval are complicated and are not shown.

The $p$-value for a hypothesis test corresponds to the sum of the probabilities of all events that are as or more extreme than the sample result. Let $X$ be a binomial random variable with parameters $n$ and $p_0$, where $\hat{p} = x/n$ and $x$ is the observed number of events. If $\hat{p} \leq p_0$, then the one-tail probability equals $P(X \leq x)$; if $\hat{p} > p_0$, then the one-tail probability equals $P(X \geq x)$. These probabilities are calculated using the approaches from Chapter~\ref{modeling}. Two-tailed probabilities are calculated by doubling the appropriate one-tailed value.

\index{data!glioblastoma}

\begin{examplewrap}
\begin{nexample}{In 2009, the FDA Oncology Drug Advisory Committee (ODAC) recommended that the drug Avastin be approved for use in glioblastoma, a form of brain cancer. Tumor shrinkage after taking a drug is called a response; out of 85 patients, 24 exhibited a response. Historically, response probabilities for brain cancer drugs were approximately 0.05, or about 5\%. Assess whether there is evidence that the response probability for Avastin is different from previous drugs. 		
}	

$H_0: p = 0.05$; $H_A: p \neq 0.05$. Let $\alpha = 0.05$. 

The independence condition is satisfied, but the success-failure condition is not, since $np_0 = (85)(0.05) = 4.25 < 5$, so this is a setting where exact binomial probabilities should be used to calculate a $p$-value.

The sample proportion $\hat{p}$ equals $x/n = 24/85 = 0.28$. Since $\hat{p} > p_0$, calculate the two-sided $p$-value from $2 \times P(X \geq 24)$, where $X \sim \text{Binom}(85, 0.05)$.

Calculating the $p$-value is best done in software; the \textsf{R} command \texttt{pbinom} returns a value of $5.3486 \times 10^{-12}$.\footnotemark{}

The $p$-value is highly significant and suggests that the response probability for Avastin is higher than for previous brain cancer drugs. The FDA staff considered this evidence sufficiently strong to justify approval for the use of the drug, even though the FDA normally requires evidence from two independently conducted randomized trials. 
\end{nexample}
\end{examplewrap}
\footnotetext{\texttt{2*pbinom(q = 23, size = 85, p = 0.05, lower.tail = FALSE)}}

\textD{\newpage}

\begin{exercisewrap}
\begin{nexercise}
Medical consultants assist patients with all aspects of an organ donation surgery, with the goal of reducing the possibility of complications during the medical procedure and recovery. To attract customers, one consultant noted that while the usual proportion of complications in liver donation surgeries in the United States is about 10\%, only 3 out of her 62 clients experienced complications with liver donor surgeries. Is there evidence to suggest that the proportion of complications in her patients is lower than the national average?\footnotemark{}
\end{nexercise}
\end{exercisewrap}
\footnotetext{Assume that the 62 patients in her dataset may be viewed as a random sample from patients receiving a donated liver. The sample proportion $\hat{p} = 3/62 = 0.048$. Under the null hypothesis, the expected number of complications is $62(0.10) = 6.2$, so the normal approximation may not be accurate and it is best to use exact binomial probabilities. Since $\hat{p} \leq p_0$, find the $p$-value by calculating $P(X \leq 3)$ when $X$ has a binomial distribution with parameters $n = 62,\, p = 0.10$: $P(X \leq 3) = 0.121$. There is not sufficient evidence to suggest that the proportion of complications among her patients is lower than the national average.}


\subsection{Choosing a sample size when estimating a proportion}

\index{margin of error|(}
\index{sample size!estimating a proportion|(}

Whenever possible, a sample size for a study should be estimated before data collection begins.  Section \ref{PowerForDifferenceOfTwoMeans} explored the calculation of sample sizes that allow a hypothesis test comparing two groups to have adequate power.  When estimating a proportion, preliminary sample size calculations are often done to estimate a sample size large enough to make the \term{margin of error} $m$ in a confidence interval sufficiently small for the interval to be useful. Recall that the margin of error $m$ is the term that is added to and subtracted from the point estimate.  Statistically, this means estimating a sample size $n$ so that the sample proportion is within some margin of error $m$ of the actual proportion with a certain level of confidence. When the normal approximation is used for a binomial proportion, a sample size sufficiently large to have a margin of error of $m$ will satisfy 
\begin{align*}
 m = (z^{\star})(\text{s.e.}(\hat{p})) = z^{\star} \sqrt{\frac{(p)(1 - p)}{n}}. 
\end{align*}
Algebra can be used to show that the above equation implies
\begin{align*}
n = \frac{(z^{\star})^2(p)(1 - p)}{m^2}.
\end{align*}
In some settings a preliminary estimate for $p$ can be used to calculate $n$.  When no estimate is available, calculus can be used to show that $p(1 - p)$ has its largest value when $p = 0.50$, and that conservative value for $p$ is often used to ensure that $n$ is sufficiently large regardless of the value of the unknown population proportion $p$.  In  that case, $n$ satisfies
\begin{align*}
  n \geq \frac{(z^{\star})^2(0.50)(1-0.50)}{m^2} = \frac{(z^{\star})^2}{4m^2}.
\end{align*}

\textD{\newpage}

\begin{examplewrap}
\begin{nexample}{Donor organs for organ transplant are scarce. Studies are conducted to explore whether the population of eligible organs can be expanded. Suppose a research team is studying the possibility of transplanting lungs from hepatitis C positive individuals; recipients can be treated with one of the new drugs that cures hepatitis C. Preliminary studies in organ transplant are often designed to estimate the probability of a successful organ graft 6 months after the transplant.  How large should a study be so that the 95\% confidence interval for the probability of a successful graft at 6 months is no wider than 20\%?}

A confidence interval no wider than 20\% has a margin of error of 10\%, or 0.10.  Using the conservative value p = 0.50,
\[n = \frac{(1.96)^2}{(4)(0.10^2)} =  96.04.\]
Sample sizes are always rounded up, so the study should have 97 patients.  

Since the study will likely yield a value $\hat{p}$ different from 0.50, the final margin of error will be smaller than $\pm 0.10$.
\end{nexample}
\end{examplewrap}

When the confidence coefficient is 95\%, 1.96 can replaced by 2 and the sample size formula reduces to 
\begin{align*}
  n = 1/m^2.
\end{align*}
This remarkably simple formula is often used by practitioners for a quick estimate of sample size.

\index{sample size!estimating a proportion|)}

\index{data!Congress approval rating|(}

\begin{exercisewrap}
\begin{nexercise}
A 2015 estimate of Congress' approval rating was 19\%.\footnotemark{} Using this estimate, how large should an additional survey be to produce a margin of error of 0.04 with 95\% confidence?\footnotemark{}
\end{nexercise}
\end{exercisewrap}
\addtocounter{footnote}{-1}%
\footnotetext{\oiRedirect{textbook-congress_at_19_in_May2015}{www.gallup.com/poll/183128/five-months-gop-congress-approval-remains-low.aspx}}%
\addtocounter{footnote}{1}%
\footnotetext{Apply the formula \\
\begin{align*}
1.96\times \sqrt{\frac{p(1-p)}{n}} \approx
1.96\times \sqrt{\frac{0.19(1-0.19)}{n}} &\leq 0.04 \qquad\to\qquad n \geq 369.5.
\end{align*}
A sample size of 370 or more would be reasonable.}

\index{data!Congress approval rating|)}
\index{margin of error|)}



%__________________
\section{Inference for the difference of two proportions}
\label{differenceOfTwoProportions}

Just as inference can be done for the difference of two population means, conclusions can also be drawn about the difference of two population proportions: $p_1 - p_2$. 

\subsection{Sampling distribution of the difference of two proportions}
\label{samplingDistributionDiffTwoProps}

\index{sampling distribution!difference of two proportions}
\index{point estimate!difference of two proportions}

The normal model can be applied to $\hat{p}_1 - \hat{p}_2$ if the sampling distribution for each sample proportion is nearly normal and if the samples are independent random samples from the relevant populations. 

\begin{onebox}{Conditions for the sampling distribution of $\hat{p}_1 - \hat{p}_2$ to be approximately normal}
The difference $\hat{p}_1 - \hat{p}_2$ tends to follow a normal model when
\begin{itemize}
\setlength{\itemsep}{0mm}
\item each of the two samples are random samples from a population,
\item the two samples are independent of each other, and
\item each sample proportion follows (approximately) a normal model. This condition is satisfied when $n_1p_1, n_1(1 - p_1), n_2 p_2$ and $n_2(1 - p_2)$ are all $\geq 10$.
\end{itemize}
The standard error of the difference in sample proportions is
\index{standard error (SE)!difference in proportions}
\begin{eqnarray}
SE_{\hat{p}_1 - \hat{p}_2}
	= \sqrt{SE_{\hat{p}_1}^2 + SE_{\hat{p}_2}^2}
	= \sqrt{\frac{p_1(1-p_1)}{n_1} + \frac{p_2(1-p_2)}{n_2}},
\label{seForDiffOfProp}
\end{eqnarray}
where $p_1$ and $p_2$ are the population proportions, and $n_1$ and $n_2$ are the two sample sizes.
\end{onebox}


\textD{\newpage}


\subsection{Confidence intervals for $\pmb{\MakeLowercase{p_1 - p_2}}$}
\label{confidenceIntervalsDifferenceProportions}
\index{confidence interval!difference of two proportions}

When calculating confidence intervals for a difference of two proportions using the normal approximation to the binomial, the two sample proportions are used to verify the success-failure condition and to compute the standard error.

\begin{examplewrap}
\begin{nexample}{The way a question is phrased can influence a person's response. For example, Pew Research Center conducted a survey with the following question:\footnotemark{}
\begin{quote}
As you may know, by 2014 nearly all Americans will be required to have health insurance. [People who do not buy insurance will pay a penalty] while [People who cannot afford it will receive financial help from the government]. Do you approve or disapprove of this policy?
\end{quote}
\index{data!health care|(}For each randomly sampled respondent, the statements in brackets were randomized: either they were kept in the order given above, or the order of the two statements was reversed. Figure~\ref{pewPollResultsForRandomizedStatementOrdering} shows the results of this experiment. Calculate and interpret a 90\% confidence interval of the difference in the probability of approval of the policy.}
First the conditions for the use of a normal model must be verified. The Pew Research Center uses sampling methods that produce random samples of the US population (at least approximately) and because each group was a simple random sample from less than 10\% of the population, the observations are independent, both within the samples and between the samples. The success-failure condition also holds for each sample, so the normal model can be used for confidence intervals for the difference in approval proportions.  The point estimate of the difference in support, where $\hat{p}_1$ corresponds to the original ordering and $\hat{p}_2$ to the reversed ordering:
$$\hat{p}_{1} - \hat{p}_{2} = 0.47 - 0.34 = 0.13.$$
The standard error can be computed from Equation~\eqref{seForDiffOfProp} using the sample proportions:
$$SE \approx \sqrt{\frac{0.47(1-0.47)}{771} + \frac{0.34(1-0.34)}{732}} = 0.025.$$
For a 90\% confidence interval, $z^{\star} = 1.65$:
$$\text{point estimate} \ \pm\ z^{\star} \times SE \quad \to \quad 0.13 \ \pm\ 1.65 \times  0.025 \quad \to \quad (0.09, 0.17).$$
With 90\% confidence, the proportion approving the 2010 health care law ranged between 9\% and 17\% depending on the phrasing of the question. The Pew Research Center interpreted this modestly large difference as an indication that for most of the public, opinions were still fluid on the health insurance mandate.  The law eventually passed as the Affordable Health Care Act (ACA).
\index{data!health care|)}
\end{nexample}
\end{examplewrap}
\footnotetext{\oiRedirect{textbook-health_care_bill_2012}{www.people-press.org/2012/03/26/public-remains-split-on-health-care-bill-opposed-to-mandate}. Sample sizes for each polling group are approximate.}

\begin{figure}[h]
\centering
\begin{tabular}{c c c c c}
	& Sample size ($n_i$) & Approve (\%)	& Disapprove (\%)	& Other \\
\hline
Original ordering & 771	& 47	& 49	& 3 \\
Reversed ordering & 732	& 34	& 63	& 3 \\
\hline
\end{tabular}
\caption{Results for a Pew Research Center poll where the ordering of two statements in a question regarding healthcare were randomized.\vspaceB{-2mm}}
\label{pewPollResultsForRandomizedStatementOrdering}
\end{figure}


\textD{\newpage}


\subsection{Hypothesis testing for $\pmb{\MakeLowercase{p_1 - p_2}}$}

%\index{hypothesis test!difference of two proportions}

Hypothesis tests for $p_1 - p_2$ are usually testing the null hypothesis of no difference between $p_1$ and $p_2$; i.e. $H_0:\,p_1 - p_2 = 0$. Under the null hypothesis, $\hat{p}_1 - \hat{p}_2$ is normally distributed with mean 0 and standard deviation $\sqrt{p(1-p)(\frac{1}{n_1} + \frac{1}{n_2})}$, where under the null hypothesis $p = p_1 = p_2$.

Since $p$ is unknown, an estimate is used to compute the standard error of $\hat{p}_1 - \hat{p}_2$; $p$ can be estimated by $\hat{p}$, the weighted average of the sample proportions $\hat{p}_1$ and $\hat{p}_2$:
\[\hat{p} = \dfrac{n_{1}\hat{p}_1 + n_{2}\hat{p}_2}{n_{1} + n_{2}} = \dfrac{x_{1} + x_{2}}{n_{1} + n_{2}}, \]
where $x_1$ is the number of observed events in the first sample and $x_2$ is the number of observed events in the second sample. This pooled proportion $\hat{p}$ is also used to check the success-failure condition.

The test statistic $z$ for testing $H_0:\, p_1 = p_2$ versus $H_A: p_1 \neq p_2$ equals:

\[z = \dfrac{\hat{p}_1 - \hat{p}_2}{\sqrt{\hat{p}(1-\hat{p})\left(\frac{1}{n_1} + \frac{1}{n_2} \right)}}. \]
 
\index{data!mammography|(}
\index{data!breast cancer|(}

\begin{examplewrap}
\begin{nexample}{The use of screening mammograms for breast cancer has been controversial for decades because the overall benefit on breast cancer mortality is uncertain.  Several large randomized studies have been conducted in an attempt to estimate the effect of mammogram screening. A 30-year study to investigate the effectiveness of mammograms versus a standard non-mammogram breast cancer exam was conducted in Canada with 89,835 female participants.\footnotemark{} During a 5-year screening period, each woman was randomized to either receive annual mammograms or standard physical exams for breast cancer.  During the 25 years following the screening period, each woman was screened for breast cancer according to the standard of care at her health care center. \vspace{3mm}

At the end of the 25 year follow-up period, 1,005 women died from breast cancer. The results by intervention are summarized in Figure~\ref{mammogramStudySummaryTable}. \vspace{3mm}

Assess whether the normal model can be used to analyze the study results.}\label{mammogramSuccessFailure}%

Since the participants were randomly assigned to each group, the groups can be treated as independent, and it is reasonable to assume independence of patients within each group.  Participants in randomized studies are rarely random samples from a population, but the investigators in the Canadian trial recruited  participants using a general publicity campaign, by sending personal invitation letters to women identified from general population lists, and through contacting family doctors.  In this study, the participants can reasonably be thought of as a random sample.

The pooled proportion $\hat{p}$ is  

\[\hat{p} = \dfrac{x_{1} + x_{2}}{n_{1} + n_{2}} = \dfrac{500 + 505}{500 + 44,425 + 505 + 44,405} = 0.0112. \]

Checking the success-failure condition for each group: 
\begin{align*}
\hat{p} \times n_{mgm} &= 0.0112 \times \text{44,925} = 503
& (1 - \hat{p}) \times n_{mgm} &= 0.9888 \times \text{44,925} = \text{44,422} \\
\hat{p} \times n_{ctrl} &= 0.0112 \times \text{44,910} = 503
& (1 - \hat{p}) \times n_{ctrl} &= 0.9888 \times \text{44,910} = \text{44,407}
\end{align*}
All values are at least 10. 

The normal model can be used to analyze the study results.
\end{nexample}
\end{examplewrap}
\footnotetext{\oiRedirect{textbook-90k_mammogram_study_2014}{Miller AB. 2014. \emph{Twenty five year follow-up for breast cancer incidence and mortality of the Canadian National Breast Screening Study: randomised screening trial}. BMJ 2014;348:g366 doi: 10.1136/bmj.g366}}

\begin{figure}[h]
	\centering
	\begin{tabular}{rrcc}
		& \multicolumn{3}{c}{Death from breast cancer?} \\
		\cline{2-4}
		& \ \hspace{3mm}\ & Yes & No \\
		\hline
		Mammogram && 500 & 44,425 \\
		Control && 505 & 44,405 \\
		\hline
	\end{tabular}
	\caption{Summary results for the mammogram study.}
	\label{mammogramStudySummaryTable}
\end{figure}

\textD{\newpage}

\begin{examplewrap}
\begin{nexample}{Do the results from the study provide convincing evidence of a difference in the proportion of breast cancer deaths between women who had annual mammograms during the screening period versus women who received annual screening with physical exams?}\label{mammogramExProp}%
The null hypothesis is that the probability of a breast cancer death is the same for the women in the two groups. If group 1 represents the mammogram group and group 2 the control group, $H_0: p_1 = p_2$ and $H_A: p_1 \neq p_2$.  Let $\alpha = 0.05$.

Calculate the test statistic $z$:
\[z = \dfrac{0.01113 - 0.01125}{\sqrt{(0.0112)(1-0.0112)\left(\frac{1}{44,925} + \frac{1}{44,910} \right)}} = -0.17.
\]	
The two-sided $p$-value is $P|Z| \ge 0.17 = 0.8650$,  which is greater than 0.05. There is insufficient evidence to reject the null hypothesis; the observed difference in breast cancer death rates is reasonably explained by chance. 	

Evaluating medical treatments typically requires accounting for additional evidence that cannot be evaluated from a statistical test. For example, if mammograms are much more expensive than a standard screening and do not offer clear benefits, there is reason to recommend standard screenings over mammograms. This study also found that a higher proportion of diagnosed breast cancer cases in the mammogram screening arm (3250 in the mammogram group vs 3133 in the physical exam group), despite the nearly equal number of breast cancer deaths.  The investigators inferred that mammograms may cause over-diagnosis of breast cancer, a phenomenon in which a breast cancer diagnosed with mammogram and subsequent biopsy may never become symptomatic. The possibility of over-diagnosis is one of the reasons mammogram screening remains controversial.
\end{nexample}
\end{examplewrap}

\begin{examplewrap}
\begin{nexample}{Calculate a 95\% confidence interval for the difference in proportions of deaths from breast cancer from the Canadian study.}\label{mammogramExConfInt}%

The independence and random sampling conditions have already been discussed.  The success failure condition should be checked for each sample, since this is not a hypothesis testing context (i.e., there is no null hypothesis). For the mammogram group, $\hat{p}_1 = 0.01113$; $n_1 \hat{p}_1 = (0.1113)(44,925) = 500$ and $n_1 (1 - \hat{p}_1) = 39,925.$ It is easy to show that the success failure condition is holds for the control group as well.

The point estimate for the difference in the probability of death is
$$\hat{p}_{1} - \hat{p}_{2} = 0.01113 - 0.01125 = -0.00012,$$ or 0.012\%.

The standard error for the estimated difference uses the individual estimates of the probability of a death:
$$SE \approx \sqrt{\frac{0.01113(1-0.01113)}{44,925} + \frac{0.01125(1-0.01125)}{44,910}} = 0.0007. $$

The 95\% confidence interval is given by 
$$ -0.00012 \pm (1.96) (0.0007) = (-0.0015, 0.0013).$$

With 95\% confidence, the difference in the probability of death is between -0.15\% and 0.13\%. As expected from the large $p$-value, the confidence interval contains the null value 0.
\end{nexample}
\end{examplewrap}

\index{data!breast cancer|)}
\index{data!mammography|)}


%__________________
\section{Inference for two or more groups}
\label{twoWayTablesAndChiSquare}

The comparison of the proportion of breast cancer deaths between the two groups can also be approached using a two-way contingency table, which contains counts for combinations of outcomes for two variables. The results for the mammogram study in this format are shown in Figure~\ref{mammogramStudySummaryTableWithTotals}.

Previously, the main question of interest was stated as, "Is there evidence of a difference in the proportion of breast cancer deaths between the two screening groups?" If the probability of a death from breast cancer does not depend the method of screening, then screening method and outcome are independent. Thus, the question can be re-phrased: "Is there evidence that screening method is associated with outcome?"

Hypothesis testing in a two-way table assesses whether the two variables of interest are associated (i.e., not independent). The approach can be applied to settings with two or more groups and for responses that have two or more categories. The observed number of counts in each table cell are compared to the number of \term{expected counts}, where the expected counts are calculated under the assumption that the null hypothesis of no association is true. A $\chi^2$ test of significance is based on the differences between observed and expected values in the cells.

\begin{figure}[h]
	\centering
	\begin{tabular}{l| l l l l |l}
		\hline
		Death from BC & \hspace{1mm}  & Yes & No & \hspace{1mm} & Total \\
		\hline
		Mammogram				   &    & 500 & 44,425 & 				&44,925 \\
		Control				   &     & 505	& 44,405    &				& 44,910 \\
		\hline
		Total						   &    & 1,005 & 88,830 & 				& 89,835 \\
		\hline
	\end{tabular}
	\caption{Results of the mammogram study, as a contingency table with marginal totals.}
	\label{mammogramStudySummaryTableWithTotals}
\end{figure}

\begin{exercisewrap}
\begin{nexercise}
Formulate hypotheses for a contingency-table approach to analyzing the mammogram data.\footnotemark{}
\end{nexercise}
\end{exercisewrap}
\footnotetext{$H_0$: There is no association between type of breast cancer screening and death from breast cancer. $H_A$: There is an association between type of breast cancer screening and death from breast cancer.}


\subsection{Expected counts}
\label{twoWayTablesExpectedCounts}

If type of breast cancer screening had no effect on outcome in the mammogram data, what would the expected results be? 

Recall that if two events $A$ and $B$ are independent, then $P(A \cap B) = P(A)P(B)$. Let $A$ represent assignment to the mammogram group and $B$ the event of death from breast cancer. Under independence, the number of individuals out of 89,835 that are expected to be in the mammogram screening group and die from breast cancer equals:
\[(89,835) P(A)P(B) = (89,835) \left(\frac{44,925}{89,835}\right) \left(\frac{1,005}{89,835} \right) = 502.6. \]

Note that the quantities 44,925 and 1,005 are the row and column totals corresponding to the upper left cell of Figure~\ref{mammogramStudySummaryTableWithTotals}, and 89,835 is the total number $n$ of observations in the table. A general formula for computing expected counts for any cell can be written from the marginal totals and the total number of observations.

\textD{\newpage}

\begin{onebox}{Computing expected counts in a two-way table}
To calculate the expected count for the $i^{th}$ row and $j^{th}$ column, compute
$$\text{Expected Count}_{\text{row }i,\text{ col }j} = \frac{(\text{row $i$ total}) \times  (\text{column $j$ total})}{\text{table total}}.\vspace{2mm}$$
\end{onebox}	
	
\begin{examplewrap}
\begin{nexample}{Calculate expected counts for the data in Figure~\ref{mammogramStudySummaryTableWithTotals}.}
\[E_{1,1} = \dfrac{44,925 \times 1,005}{89,835} = 502.6 \qquad E_{1,2} = \dfrac{44,925 \times 88,830}{89,835} = 44,422.4\]
\[E_{2,1} = \dfrac{2,922 \times 1,005}{89,835} = 502.4 \qquad E_{2,2} = \dfrac{7,078 \times 88,830}{89,835} = 44,407.6\]
\end{nexample}
\end{examplewrap}

\begin{figure}[h]
	\centering
		\begin{tabular}{l| l l l l| l}
			\hline
			Death from BC & \hspace{1mm}  & Yes & No & \hspace{1mm} & Total \\
			\hline
			Mammogram				   &    & 500 \highlightO{(502.6)} & 44,425  \highlightO{(44,422.4)} & 				&44,925 \\
			Control				   &     & 505  \highlightO{(502.4)}	& 44,405  \highlightO{(44,407.6)}  &				& 44,910 \\
			\hline
			Total						   &    & 1,005 & 88,830 & 				& 89,835 \\
			\hline
		\end{tabular}
	\caption{Results of the mammogram study, with \highlightO{(expected counts)}. The expected counts should also sum to the row and column totals; this can be a useful check for accuracy.}
	\label{mammogramStudyExpectedCounts}
\end{figure}

\index{data!hiv|(}

\begin{examplewrap}
\begin{nexample}{If a newborn is HIV$^+$, should he or she be treated with nevirapine (NVP) or a more expensive drug, lopinarvir (LPV)? In this setting, success means preventing virologic failure; i.e., growth of the virus. A randomized study was conducted to assess whether there is an association between treatment and outcome.\footnotemark{} Of the 147 children administered NVP, about 41\% experienced virologic failure; of the 140 children administered LPV, about 19\% experienced virologic failure. Construct a table of observed counts and a table of expected counts.}
	
Convert the proportions to count data: 41\% of 147 is approximately 60, and 19\% of 140 is approximately 27. The observed results are given in Figure~\ref{violariHivStudyObsCounts}.

Calculate the expected counts for each cell:
\[E_{1, 1} = \dfrac{87 \times 147}{287} = 44.6 \qquad E_{1, 2} = \dfrac{87 \times 140}{287} = 42.4 \]
\[E_{2, 1} = \dfrac{200 \times 147}{287} = 102.4 \qquad E_{2, 2} = \dfrac{200 \times 140}{287} = 97.6 \]

The expected counts are summarized in Figure~\ref{violariHivStudyExpCounts}.
\end{nexample}
\end{examplewrap}
\footnotetext{Violari A, et al. N Engl J Med 2012; 366:2380-2389
DOI: 10.1056/NEJMoa1113249}

\begin{figure}[h]
	\centering
	\begin{tabular}{l | l l | l}
	\hline
	& NVP & LPV & Total \\
	\hline
	Virologic Failure & 60 & 27 & 87 \\
	Stable Disease & 87 & 113 & 200 \\	
	\hline
	Total & 147 & 140 & 287 \\
	\hline
	\end{tabular}
	\caption{Observed counts for the HIV study.}
	\label{violariHivStudyObsCounts}
\end{figure}

\begin{figure}[h]
	\centering
	\begin{tabular}{l | l l | l}
		\hline
		& NVP & LPV & Total \\
		\hline
		Virologic Failure & 44.6 & 42.4 & 87 \\
		Stable Disease & 102.4 & 97.6 & 200 \\
		\hline	
		Total & 147 & 140 & 287 \\
		\hline
	\end{tabular}
	\caption{Expected counts for the HIV study.}
	\label{violariHivStudyExpCounts}
\end{figure}

\index{data!hiv|)}


\textD{\newpage}


\subsection{The $\pmb{\chi^2}$ test statistic}

Previously, test statistics have been constructed by calculating the difference between a point estimate and a null value, then dividing by the standard error of the point estimate to standardize the difference. The $\chi^2$ statistic is based on a different idea.  In each cell of a table, the difference \emph{observed} - \emph{expected} is a measure of the discrepancy between what was observed in the data and what should have been observed under the null hypothesis of no association. If the row and column variables are highly associated, that difference will be large.  Two adjustments are made to the differences before the final statistic is calculated.  First, since both positive and negative differences suggest a lack of independence, the differences are squared to remove the effect of the sign.  Second, cells with larger counts may have larger discrepancies by chance alone, so the squared differences in each cell are scaled by the number expected in the cell under the hypothesis of independence.  The final $\chi^2$ statistic is the sum of these standardized squared differences, where the sum has one term for each cell in the table.

The $\chi^2$ test statistic\marginpar[\raggedright\vspace{9mm}

$\chi^2$\vspace{0.5mm}\\\footnotesize chi-square\\test statistic]{\raggedright\vspace{9mm}
	
	$\chi^2$\vspace{0.5mm}\\\footnotesize chi-square\\test statistic}\index{chi-square statistic} is calculated as:

\[\chi^2 = \sum_{\text{all cells}} \frac{(\text{observed} - \text{expected})^2}{\text{expected}}. \]

The theory behind the $\chi^2$ test and its sampling distribution relies on the same normal approximation to the binomial distribution that was introduced earlier.  The cases in the dataset must be independent and each expected cell count should be at least 10.  The second condition can be relaxed in tables with more than 4 cells.

\begin{onebox}{Conditions for the $\pmb{\chi^2}$ test}
Two conditions that must be checked before performing a $\chi^2$ test:
\begin{description}
\setlength{\itemsep}{0mm}
	\item[Independence.] Each case that contributes a count to the table must be independent of all the other cases in the table.
	\item[Sample size.] Each expected cell count must be greater than or equal to 10. For tables larger than $2 \times 2$, it is appropriate to use the test if no more than 1/5 of the expected counts are less than 5, and all expected counts are greater than 1.
\end{description}
\end{onebox}


\begin{examplewrap}
\begin{nexample}{For the mammogram data, check the conditions for the $\chi^2$ test and calculate the $\chi^2$ test statistic.}

Independence is a reasonable assumption, since individuals have been randomized to either the treatment or control group. Each expected cell count is greater than 10.
\begin{align*}
\chi^2 &= \sum_{\text{all cells}} \frac{(\text{observed} - \text{expected})^2}{\text{expected}} \\
&= \dfrac{(500 - 502.6)^2}{502.6} + \dfrac{(44,425 - 44,422.4)^2}{44,422.4} + \dfrac{(505 - 502.4)^2}{502.4} + \dfrac{(44,405 - 44,407.6)^2}{44,407.6} \\
&=0.02.
\end{align*}	
\end{nexample}
\end{examplewrap}

\begin{exercisewrap}
\begin{nexercise}
For the HIV data, check the conditions for the $\chi^2$ test and calculate the $\chi^2$ test statistic.\footnotemark{}
\end{nexercise}
\end{exercisewrap}
\footnotetext{Independence holds, since this is a randomized study. The expected counts are greater than 10. $\chi^2 = \frac{(60-44.6)^2}{44.6} + \frac{(27-42.4)^2}{42.4} + \frac{(87-102.4)^2}{102.4} + \frac{(113-97.6)^2}{97.6} = 14.7.$}


\subsection{Calculating $\pmb{\MakeLowercase{p}}$-values for a $\pmb{\chi^2}$ distribution}

The \term{chi-square distribution} is often used with data and statistics that are positive and right-skewed.  The distribution is characterized by a single parameter, the degrees of freedom. Figure~\ref{chiSquareDistributionWithInceasingDF} demonstrates three general properties of chi-square distributions as the degrees of freedom increases: the distribution becomes more symmetric, the center moves to the right, and the variability increases.

\begin{figure}[h]
	\centering
	\includegraphics[width=0.9\textwidth]{ch_inference_for_props_oi_biostat/figures/chiSquareDistributionWithInceasingDF/chiSquareDistributionWithInceasingDF}
	\caption{Three chi-square distributions with varying degrees of freedom.}
	\label{chiSquareDistributionWithInceasingDF}
\end{figure}

\textD{\newpage}

The $\chi^2$ statistic from a contingency table has a sampling distribution that approximately follows a $\chi^2$ distribution with degrees of freedom $df = (r-1)(c-1)$, where $r$ is the number of rows and $c$ is the number of columns. Either statistical software or a table can be used to calculate $p$-values from the $\chi^2$ distribution. The \term{chi-square table} is partially shown in Figure~\ref{chiSquareProbabilityTableShort}, and a more complete table is presented in Appendix~\vref{chiSquareProbabilityTable}. This table is very similar to the $t$-table: each row provides values for distributions with different degrees of freedom, and a cut-off value is provided for specified tail areas. One important difference from the $t$-table is that the $\chi^2$ table only provides upper tail values.

\begin{figure}[h]
	\centering
	\begin{tabular}{r | rrrr | rrrr |}
		\hline
		Upper tail & 0.3 & 0.2 & 0.1 & 0.05 & 0.02 & 0.01 & 0.005 & 0.001 \\ 
		\hline
		df \hfill 1 & \footnotesize 1.07 & \footnotesize 1.64 & \footnotesize 2.71 & \footnotesize 3.84 & \footnotesize 5.41 & \footnotesize 6.63 & \footnotesize 7.88 & \footnotesize 10.83 \\ 
		\hfill 2 & \footnotesize 2.41 & \footnotesize 3.22 & \footnotesize 4.61 & \footnotesize 5.99 & \footnotesize 7.82 & \footnotesize 9.21 & \footnotesize 10.60 & \footnotesize 13.82 \\ 
		3 & \footnotesize 3.66 & \footnotesize 4.64 & \footnotesize 6.25 & \footnotesize 7.81 & \footnotesize 9.84 & \footnotesize 11.34 & \footnotesize 12.84 & \footnotesize 16.27 \\ 
		4 & \footnotesize 4.88 & \footnotesize 5.99 & \footnotesize 7.78 & \footnotesize 9.49 & \footnotesize 11.67 & \footnotesize 13.28 & \footnotesize 14.86 & \footnotesize 18.47 \\ 
		5 & \footnotesize 6.06 & \footnotesize 7.29 & \footnotesize 9.24 & \footnotesize 11.07 & \footnotesize 13.39 & \footnotesize 15.09 & \footnotesize 16.75 & \footnotesize 20.52 \\ 
		\hline
		6 & \footnotesize 7.23 & \footnotesize 8.56 & \footnotesize 10.64 & \footnotesize 12.59 & \footnotesize 15.03 & \footnotesize 16.81 & \footnotesize 18.55 & \footnotesize 22.46 \\ 
		7 & \footnotesize 8.38 & \footnotesize 9.80 & \footnotesize 12.02 & \footnotesize 14.07 & \footnotesize 16.62 & \footnotesize 18.48 & \footnotesize 20.28 & \footnotesize 24.32 \\ 
		\hline
	\end{tabular}
	\caption{A section of the chi-square table. A complete table is in Appendix~\vref{chiSquareProbabilityTable}.}
	\label{chiSquareProbabilityTableShort}
\end{figure}

\begin{examplewrap}
\begin{nexample}{Calculate an approximate $p$-value for the mammogram data, given that the $\chi^2$ statistic equals 0.02. Assess whether the data provides convincing evidence of an association between screening group and breast cancer death.}

The degrees of freedom in a $2 \times 2$ table is 1, so refer to the values in the first column of the probability table. The value 0.02 is less than 1.07, so the $p$-value is greater than 0.3. The data do not provide convincing evidence of an association between screening group and breast cancer death. This supports the conclusions from Example~\ref{mammogramExProp}, where the $p$-value was calculated to be 0.8650 and is visualized in Figure~\ref{mammogramPValue}.
\end{nexample}
\end{examplewrap}

\begin{figure}[h]
	\centering
	\includegraphics[width=0.6\textwidth]{ch_inference_for_props_oi_biostat/figures/mammogramPValue/mammogramPValue}
	\caption{The $p$-value for the mammogram data is shaded on the $\chi^2$ distribution with $df=1$.The shaded area is to the right of x = 0.02.}
	\label{mammogramPValue}
\end{figure}

\begin{exercisewrap}
\begin{nexercise}\label{hivDataPValue}%
Calculate an approximate $p$-value for the HIV data. Assess whether the data provides convincing evidence of an association between treatment and outcome at the $\alpha = 0.01$ significance level.\footnotemark{}
\end{nexercise}
\end{exercisewrap}
\footnotetext{The $\chi^2$ statistic is 14.7. For degrees of freedom 1, the tail area beyond 14.7 is smaller than 0.001. There is evidence to suggest that treatment is not independent of outcome.}


\textD{\newpage}


\subsection{Interpreting the results of a $\pmb{\chi^2}$ test}

If the $p$-value from a $\chi^2$ test is small enough to provide evidence to reject the null hypothesis of no association, it is important to explore the results further to understand direction of the observed association. This is done by examining the residuals, the standardized differences of the \emph{observed} - \emph{expected}, for each cell. Instead of using squared differences, the residuals are based on the differences themselves, and the standardizing or scaling factor is $\sqrt{\text{expected}}$.  Calculating residuals can be particularly helpful for understanding the results from large tables.

\index{residuals!contingency table}

For each cell in a table, the residual equals:
\[\dfrac{\text{observed} - \text{expected}}{\sqrt{\text{expected}}}. \]
Residuals with a large magnitude contribute the most to the $\chi^2$ statistic. If a residual is positive, the observed value is greater than the expected value, and vice versa for a negative residual.

\index{data!famuss|(}

\begin{examplewrap}
  \begin{nexample}
    {In the FAMuSS study introduced in Chapter~\ref{introductionToData}, researchers measured a variety of demographic and genetic characteristics for about 1,300 participants, including data on race and genotype at a specific locus on the ACTN3 gene (Figure~\ref{famussObservedCountsRaceGenotype}).
    Is there evidence of an association between genotype and race?} First, check the assumptions for applying a $\chi^2$ test. It is reasonable to assume independence, since it is unlikely that any participants were related to each other. None of the expected counts, as shown in Figure~\ref{famussExpectedRaceGenotype}, are less than 5.

$H_0$: Race and genotype are independent.

$H_A$: Race and genotype are not independent.

Let $\alpha = 0.05$.

Calculate the $\chi^2$ statistic:
\begin{align*}
\chi^2 &= \sum_{\text{all cells}} \frac{(\text{observed} - \text{expected})^2}{\text{expected}} \\
&= \dfrac{(16-7.85)^2}{7.85} + \dfrac{(6-11.84)^2}{11.84} + ... + \dfrac{(5 - 6.22)^2}{6.22} \\
&=19.4.
\end{align*}	

Calculate the $p$-value: for a table with 3 rows and 5 columns, the $\chi^2$ statistic is distributed with $(3-1)(5-1) = 8$ degrees of freedom. From the table, a $\chi^2$ value of 19.4 corresponds to a tail area between 0.01 and 0.02. Thus, there is sufficient evidence to reject the null hypothesis of independence between race and genotype.

The $p$-value can be obtained using the \textsf{R} function \texttt{pchisq} (\texttt{pchisq(19.4, df = 8, lower.tail = FALSE)}), which returns a value of 0.012861. %\footnote{\texttt{pchisq(19.4, df = 8, lower.tail = FALSE)}}

To further explore the differences in genotype distribution between races, calculate residuals for each cell (Figure~\ref{famussResidualsRaceGenotype}). The largest residuals are in the first row; there are many more African Americans with the CC genotype than expected under independence, and fewer with the CT genotype than expected. The residuals in the second row indicate a similar trend for Asians, but with a less pronounced difference. These results suggest further directions for research; a future study could enroll a larger number of African American and Asian participants to examine whether the observed trend holds with a more representative sample. Geneticists might also be interested in exploring whether this genetic difference between populations has an observable phenotypic \mbox{effect}.
\end{nexample}
\end{examplewrap}

% latex table generated in R 3.2.4 by xtable 1.8-2 package
% Wed Dec 07 22:13:57 2016
\begin{figure}[ht]
	\centering
	\begin{tabular}{r | l l l | l}
		\hline
		& CC & CT & TT & Sum \\ 
		\hline
		African American & 16 & 6 & 5 & 27 \\ 
		Asian & 21 & 18 & 16 & 55 \\ 
		Caucasian & 125 & 216 & 126 & 467 \\ 
		Hispanic & 4 & 10 & 9 & 23 \\ 
		Other & 7 & 11 & 5 & 23 \\ 
		\hline
		Sum & 173 & 261 & 161 & 595 \\ 
		\hline
	\end{tabular}
	\caption{Observed counts for race and genotype data from the FAMuSS study.}
    \label{famussObservedCountsRaceGenotype}
\end{figure}
%labels modified from xtable

% latex table generated in R 3.2.4 by xtable 1.8-2 package
% Thu Dec 08 10:34:51 2016
\begin{figure}[ht]
	\centering
	\begin{tabular}{r| l l l | l}
		\hline
		& CC & CT & TT & Sum \\ 
		\hline
		African Am & 7.85 & 11.84 & 7.31 & 27.00 \\ 
		Asian & 15.99 & 24.13 & 14.88 & 55.00 \\ 
		Caucasian & 135.78 & 204.85 & 126.36 & 467.00 \\ 
		Hispanic & 6.69 & 10.09 & 6.22 & 23.00 \\ 
		Other & 6.69 & 10.09 & 6.22 & 23.00 \\ 
		\hline
		Sum & 173.00 & 261.00 & 161.00 & 595.00 \\ 
		\hline
	\end{tabular}
	\caption{Expected counts for race and genotype data from the FAMuSS study.}
	\label{famussExpectedRaceGenotype}
\end{figure}
%labels and formatting modified

% latex table generated in R 3.2.4 by xtable 1.8-2 package
% Thu Dec 08 10:48:57 2016
\begin{figure}[ht]
	\centering
	\begin{tabular}{r|lll|l}
		\hline
		& CC & CT & TT & Sum \\ 
		\hline
		African Am & \highlightO{2.91} & \highlightO{-1.70} & -0.85 & 0.00 \\ 
		Asian & \highlightO{1.25} & \highlightO{-1.25} & 0.29 & 0.00 \\ 
		Caucasian & -0.93 & 0.78 & -0.03 & 0.00 \\ 
		Hispanic & -1.04 & -0.03 & 1.11 & 0.00 \\ 
		Other & 0.12 & 0.29 & -0.49 & 0.00 \\ 
		\hline
		Sum & 0.00 & 0.00 & 0.00 & 0.00 \\ 
		\hline
	\end{tabular}
	\caption{Residuals for race and genotype data from the FAMuSS study.}
     \label{famussResidualsRaceGenotype}
\end{figure}

\index{data!famuss|)}
\index{data!hiv}

\begin{examplewrap}
\begin{nexample}{In Guided Practice~\ref{hivDataPValue}, the $p$-value was found to be smaller than 0.001, suggesting that treatment is not independent of outcome. Does the evidence suggest that infants should be given nevirapine or lopinarvir?}\label{HIVDirectionEx}%

In a $2 \times 2$ table, it is relatively easy to directly compare observed and expected counts. 
For nevirapine, more infants than expected experienced virologic failure (60 > 44.6), while fewer than expected reached a stable disease state (87 < 102.4). For lopinarvir, fewer infants than expected experienced virologic failure (27 < 42.4), and more infants than expected reached a stable disease state (113 > 97.6) (Figure~\ref{observedAndExpectedCountsForTheHivStudy}). The outcomes for infants on lopinarvir are better than for those on nevirapine; combined with the results of the significance test, the data suggest that lopinarvir is associated with better treatment outcomes.
\end{nexample}
\end{examplewrap}

\begin{figure}[h]
	\centering
	\begin{tabular}{l | l l | l}
		\hline
		& NVP & LPV & Total \\
		\hline
		Virologic Failure & 60 \highlightO{44.6} & 27 \highlightO{42.4} & 87 \\
		Stable Disease & 87 \highlightO{102.4} & 113 \highlightO{97.6}& 200 \\	
		\hline
		Total & 147 & 140 & 287 \\
		\hline
	\end{tabular}
	\caption{Observed and \highlightO{(expected)} counts for the HIV study.}
	\label{observedAndExpectedCountsForTheHivStudy}
\end{figure}			
		
\begin{exercisewrap}
\begin{nexercise}
Confirm the conclusions reached in Example~\ref{HIVDirectionEx} by analyzing the residuals.\footnotemark{}
\end{nexercise}
\end{exercisewrap}
\footnotetext{$R_{1, 1} = \frac{(44.6-60)}{\sqrt{44.6}} = 2.31$; $R_{1, 2} = \frac{(42.4-27)}{\sqrt{27}} = -2.37$; $R_{2, 1} = \frac{(87-102.4)}{\sqrt{102.4}} = -1.53$; $R_{2, 2} = \frac{(113-97.6)}{\sqrt{97.6}} = 1.56$. The positive residuals for the upper left and lower right cells indicate that more infants than expected experienced virologic failure on NVP and stable disease on LPV; vice versa for the upper right and lower left cells. The larger magnitude of the residuals for the two NVP cells indicates that most of the discrepancy between observed and expected counts is for outcomes related to NVP.}

\index{data!LEAP}

\begin{exercisewrap}
\begin{nexercise}
Chapter~\ref{introductionToData} started with the discussion of a study examining whether exposure to peanut products reduce the rate of a child developing peanut allergies. Children were randomized either to the peanut avoidance or the peanut consumption group; at 5 years of age, each child was tested for peanut allergy using an oral food challenge (OFC). The results of the OFC are reproduced in Figure~\ref{leapStudyResultsTest}; failing the food challenge indicates an allergic reaction. Assess whether there is evidence for exposure to peanut allergy reducing the chance of developing peanut allergies.\footnotemark{}
\end{nexercise}
\end{exercisewrap}
\footnotetext{The assumptions for conducting a $\chi^2$ test are satisfied. Calculate a $\chi^2$ test statistic: 24.29. The associated $p$-value is $8.3 \times 10^{-7}$. There is evidence to suggest that treatment group is not independent of outcome. Specifically, a residual analysis shows that in the peanut avoidance group, more children than expected failed the OFC; in the peanut consumption group, more children than expected passed the OFC.}

% latex table generated in R 3.1.1 by xtable 1.7-4 package
% Thu Jul 16 07:12:04 2015
\begin{figure}[h]
	\centering
	\begin{tabular}{rrrr}
		\hline
		& FAIL OFC & PASS OFC & Sum \\ 
		\hline
		Peanut Avoidance & 36 & 227 & 263 \\ 
		Peanut Consumption & 5 & 262 & 267 \\ 
		Sum & 41 & 489 & 530 \\ 
		\hline
	\end{tabular}
	\caption{LEAP Study Results.} 
	\label{leapStudyResultsTest}
\end{figure}
%library(xtable); outcome.table = addmargins(table(LEAP$treatment.group, LEAP$overall.V60.outcome)); xtable(outcome.table, digits = 0, caption = "LEAP Study Results", caption  = "leapStudyResults")
	

\subsection{Fisher's exact test}

If sample sizes are too small, the $\chi^2$ distribution does not yield accurate $p$-values for  assessing  independence of the row and column variables in a table.  When expected counts in a table are less than 10, \term{Fisher's exact test} is often  used to calculate exact levels of significance. This test is usually applied to $2 \times 2$ tables. It can be applied to larger tables, but the logic behind the test is complex and the calculations involved are computationally intensive,  so this section covers only $2 \times 2$ tables. 

\index{data!fecal infusion|(}

\textit{Clostridium difficile} is a bacterium that causes inflammation of the colon. Antibiotic treatment is typically not effective, particularly for patients who experience multiple recurrences of infection. Infusion of feces from healthy donors has been reported as an effective treatment for recurrent infection. A randomized trial was conducted to compare the efficacy of donor-feces infusion versus vancomycin, the antibiotic typically prescribed to treat \textit{C. difficile }infection. The results of the trial are shown in Figure~\ref{fecalStudyResultsTest}.\footnote{These results correspond to the number of patients cured after the first infusion of donor feces and the number of patients cured in the vancomycin-alone group.} A brief calculation shows that all of the expected cell counts are less than 10, so the $\chi^2$ test should not be used as a test for association. 

Under the null hypothesis, the probabilities of cure in the fecal infusion and vancomycin groups are equal; i.e., individuals in one group are just as likely to be cured as individuals in the other group. Suppose the probability that an individual is cured, given that he or she was assigned to the fecal infusion group, is $p_1$ and the probability an individual is cured in the vancomycin group is $p_2$. Researchers were interested in testing the null hypothesis $H_0$: $p_1 = p_2$.

%data from \medicine\feces_infusion

\begin{figure}[h]
	\centering
	\begin{tabular}{rrrr}
		\hline
		& Cured & Uncured & Sum \\ 
		\hline
		Fecal Infusion & 13 & 3 & 16 \\ 
		Vancomycin & 4 & 9 & 13 \\ 
		Sum & 17 & 12 & 29 \\ 
		\hline
	\end{tabular}
	\caption{Fecal Infusion Study Results.} 
	\label{fecalStudyResultsTest}
\end{figure}

\textD{\newpage}

The $p$-value is the probability of observing results as or more extreme than those observed in the study under  the assumption that the null hypothesis is true. Previously discussed methods for significance testing have relied on calculating a test statistic associated with a defined sampling distribution, then obtaining $p$-values from tail areas on the distribution.  Fisher's exact test uses a similar approach, but introduces a new sampling distribution.  

The $p$-value for Fisher's exact test is calculated by adding together the individual conditional probabilities of obtaining each table that is as or more extreme than the one observed, under the null hypothesis and given that the marginal totals are considered fixed.

\begin{itemize}
	
	\item When the row and column totals are held constant, the value of any one cell in the table determines the rest of the entries. For example, if the marginal sums in Figure~\ref{fecalStudyResultsTest} are known, along with the value in one cell (e.g., the upper right equals 3), it is possible to calculate the values in the other three cells. Thus, when marginal totals are considered fixed, each table represents a unique set of results.
	
	\item Extreme tables are those which contradict the null hypothesis of $p_1 = p_2$. In the fecal infusion group, under the null hypothesis of no difference in the population proportion cured, one would expect $\frac{16 \times 17}{29} = 9.38$ cured individuals. The 13 observed cured individuals is extreme in the direction of more being cured than expected under the null hypothesis. An extreme result in the other direction would be, for instance, 1 cured patient in the fecal infusion group and 16 in the vancomycin group.
	
\end{itemize}

\begin{examplewrap}
\begin{nexample}{Of the 17 patients cured, 13 were in the fecal infusion group and 4 were in the vancomycin group. Assume that the marginal totals are fixed (i.e., 17 patients were cured, 12 were uncured, and 16 patients were in the fecal infusion group, while 13 were in the vancomycin group). Enumerate all possible sets of results that are more extreme than what was observed, in the same direction.}\label{fecalStudyExtreme}%
	
The observed results show a case of $\hat{p}_1 > \hat{p}_2$; results that are more extreme consist of cases where more than 13 cured patients were in the fecal infusion group. Under the assumption that the total number of cured patients is constant at 17 and that only 16 patients were assigned to the fecal infusion group (out of 29 patients total), more extreme results are represented by cases where 14, 15, or 16 cured patients were in the fecal infusion group. The following tables illustrate the unique combinations of values for the 4 table cells corresponding to those extreme results.

\begin{center}
	\color{gray}
	\begin{tabular}{r|cc|c}
		\hline
		& Cured & Uncured & Sum \\ 
		\hline
		Fecal Infusion & \textcolor{black}{14} & \textcolor{black}{2} & 16 \\ 
		Vancomycin & \textcolor{black}{3} & \textcolor{black}{10} & 13 \\ 
		\hline
		Sum & 17 & 12 & 29 \\ 
		\hline
	\end{tabular}

	\color{gray}
	\begin{tabular}{r|cc|c}
		\hline
		& Cured & Uncured & Sum \\ 
		\hline
		Fecal Infusion & \textcolor{black}{15} & \textcolor{black}{1} & 16 \\ 
		Vancomycin & \textcolor{black}{2} & \textcolor{black}{11} & 13 \\ 
		\hline
		Sum & 17 & 12 & 29 \\ 
		\hline
	\end{tabular}

	\color{gray}
	\begin{tabular}{r|cc|c}
		\hline
		& Cured & Uncured & Sum \\ 
		\hline
		Fecal Infusion & \textcolor{black}{16} & \textcolor{black}{0} & 16 \\ 
		Vancomycin & \textcolor{black}{1} & \textcolor{black}{12} & 13 \\ 
		\hline
		Sum & 17 & 12 & 29 \\ 
		\hline
	\end{tabular}
\end{center}
\end{nexample}
\end{examplewrap}


\subsubsection{Calculating a one-sided $\pmb{\MakeLowercase{p}}$-value}

Suppose that researchers were interested in testing the null hypothesis against the one-sided alternative, $H_A: p_1 > p_2$. To calculate the one-sided $p$-value, sum the probabilities of each table representing results as or more extreme than those observed; specifically, sum the probabilities of observing Figure~\ref{fecalStudyResultsTest} and the tables in Example~\ref{fecalStudyExtreme}.

\begin{figure}[h]
	\centering
	\begin{tabular}{rccc}
		\hline
		& Cured & Uncured & Sum \\ 
		\hline
		Fecal Infusion & $a$ & $b$ & $a+b$ \\ 
		Vancomycin & $c$ & $d$ & $c+d$ \\ 
		Sum & $a+c$ & $b+d$ & $n$ \\ 
		\hline
	\end{tabular}
	\caption{General Layout of Data in Fecal Infusion Study.} 
	\label{fecalStudyGeneral}
\end{figure}

The probability of observing a table with cells $a, b, c, d$ given fixed marginal totals $a+b$, $c+d$, $a + c$, and $b +d$ follows the hypergeometric distribution.  The hypergeometric distribution was introduced in Section~\ref{hypergeometric}.

\[P(a,b,c,d) = \text{HGeom}(a+b, c+d, a+c) = \dfrac{ {a+b \choose a} {c+d \choose c}}{{n \choose a+c}} = \dfrac{(a+b)! \text{ } (c+d)! \text{ } (a+c)! \text{ } (b+d)!}{a! \text{ } b! \text{ } c! \text{ } d! \text{ } n!}.\]

\begin{examplewrap}
\begin{nexample}{Calculate the probability of observing Figure~\ref{fecalStudyResultsTest}, assuming the margin totals are fixed.}

\[P(13, 3, 4, 9) = \dfrac{ {16 \choose 13} {13 \choose 4}}{{29 \choose 17}} = \dfrac{16! \text{ } 13! \text{ } 17! \text{ } 12!}{13! \text{ } 3! \text{ } 4! \text{ } 9! \text{ } 29!} = 7.71 \times 10^{-3}.\]

The value 0.0077 represents the probability of observing 13 cured patients out of 16 individuals in the fecal infusion group and 1 cured in the vancomycin group, given that there are a total of 29 patients and 17 were cured overall.
\end{nexample}
\end{examplewrap}

\begin{examplewrap}
\begin{nexample}{Evaluate the statistical significance of the observed data in Figure~\ref{fecalStudyResultsTest} using the one-sided alternative $H_A: p_1 > p_2$.}

Calculate the probability of the tables from Example~\ref{fecalStudyExtreme}. Generally, the formula for these tables is 
\[P(a, b, c, d) = \dfrac{ {a+b \choose a} {c+d \choose c}}{{n \choose a+c}} = \dfrac{ {16 \choose a} {13 \choose c}}{{29 \choose 17}},\] 
since the marginal totals from Figure~\ref{fecalStudyResultsTest} are fixed. The value $a$ ranges from 14, 15, 16, while $c$ ranges from 3, 2, 1.
\begin{align*}
P(14, 2, 3, 10) &= \dfrac{ {16 \choose 14} {13 \choose 3}}{{29 \choose 17}} = 6.61 \times 10^{-4} \\
P(15, 1, 2, 11) &= \dfrac{ {16 \choose 15} {13 \choose 2}}{{29 \choose 17}} = 2.40 \times 10^{-5} \\
P(16, 0, 1, 12) &= \dfrac{ {16 \choose 16} {13 \choose 1}}{{29 \choose 17}} = 2.51 \times 10^{-7}
\end{align*}

The probability of the observed table is $7.71 \times 10^{-3}$, as calculated in the previous example.

The one-sided $p$-value is the sum of these table probabilities: $(7.71 \times 10^{-3}) + (6.61 \times 10^{-4}) + (2.40 \times 10^{-5}) + (2.51 \times 10^{-7}) = 0.0084.$

The results are significant at the $\alpha = 0.05$ significance level. There is evidence to support the one-sided alternative that the proportion of cured patients in the fecal infusion group is higher than the proportion of cured patients in the vancomycin group. However, it is important to note that two-sided alternatives are the standard in medical literature. Conducting a two-sided test would be especially desirable when evaluating a treatment which lacks randomized trials supporting its efficacy, such as donor-feces infusion.
\end{nexample}
\end{examplewrap}


\subsubsection{Calculating a two-sided $\pmb{\MakeLowercase{p}}$-value}

There are various methods for calculating a two-sided $p$-value in the Fisher's exact test setting. When the test is calculated by hand, the most common way to calculate a two-sided $p$-value is to double the smaller of the one-sided $p$-values. One other common method used by various statistical computing packages such as \textsf{R} is to classify "more extreme" tables as all tables with probabilities less than that of the observed table, in both directions. The two-sided $p$-value is the sum of probabilities for the qualifying tables.  That approach is illustrated in the next example.

\begin{examplewrap}
\begin{nexample}{Evaluate the statistical significance of the observed data in Figure~\ref{fecalStudyResultsTest} using the two-sided alternative $H_A: p_1 \neq p_2$.}
	
Identify tables that are more extreme in the other direction of the observed result, i.e. where the proportion of cured patients in the vancomycin group are higher than in the fecal infusion group. Start with the most extreme cases and calculate probabilities until a table has a $p$-value higher than $7.71 \times 10^{-3}$, the probability of the observed table. 

The most extreme result in the $\hat{p}_1 < \hat{p}_2$ direction would be if all patients in the vancomycin group were cured; then 13 of the cured patients would be in the vancomycin group and 4 would be in the fecal transplant group. This table has probability $3.5 \times 10^{-5}$.

\begin{center}
	\color{gray}
	\begin{tabular}{r|cc|c}
		\hline
		& Cured & Uncured & Sum \\ 
		\hline
		Fecal Infusion & \textcolor{black}{4} & \textcolor{black}{12} & 16 \\ 
		Vancomycin & \textcolor{black}{13} & \textcolor{black}{0} & 13 \\ 
		\hline
		Sum & 17 & 12 & 29 \\ 
		\hline
	\end{tabular}
\end{center}

Continue enumerating tables by decreasing the number of cured patients in the vancomycin group. The table with 5 cured patients in the fecal infusion group has probability $1.09 \times 10^{-3}$.

\begin{center}
	\color{gray}
	\begin{tabular}{r|cc|c}
		\hline
		& Cured & Uncured & Sum \\ 
		\hline
		Fecal Infusion & \textcolor{black}{5} & \textcolor{black}{11} & 16 \\ 
		Vancomycin & \textcolor{black}{12} & \textcolor{black}{1} & 13 \\ 
		\hline
		Sum & 17 & 12 & 29 \\ 
		\hline
	\end{tabular}
\end{center}

The table with 6 cured patients in the fecal infusion group has probability 0.012. This value is greater than $7.71 \times 10^{-3}$, so it will not be part of the sum to calculate the two-sided $p$-value.
	
\begin{center}
	\color{gray}
	\begin{tabular}{r|cc|c}
		\hline
		& Cured & Uncured & Sum \\ 
		\hline
		Fecal Infusion & \textcolor{black}{6} & \textcolor{black}{10} & 16 \\ 
		Vancomycin & \textcolor{black}{11} & \textcolor{black}{2} & 13 \\ 
		\hline
		Sum & 17 & 12 & 29 \\ 
		\hline
	\end{tabular}
\end{center}

As calculated in the previous example, the one-sided $p$-value is $0.0084$. Thus, the two-sided $p$-value for these data equals $0.0084 + (3.5 \times 10^{-5}) + (1.09 \times 10^{-3}) = 0.0095$. The results are significant at the $\alpha = 0.01$ significance level, and there is evidence to support the efficacy of donor-feces infusion as a treatment for recurrent \textit{C. difficile} infection.
\end{nexample}
\end{examplewrap}

\index{data!fecal infusion|)}


%__________________
\section[Chi-square tests for the fit of a distribution]{Chi-square tests for the fit of a distribution}
\label{oneWayChiSquare}

The $\chi^2$ test can also be used to examine the appropriateness of hypothesized distribution for a dataset, most commonly when a set of observations falls naturally into categories as in the examples discussed in this section. As with testing in the two-way table setting, expected counts are calculated based on the assumption that the hypothesized distribution is correct, and the statistic is based on the discrepancies between observed and expected counts. The  $\chi^2$ sampling distribution for the test statistic is reasonably accurate when each expected count is at least 5 and follows a $\chi^2$ distribution with $k-1$ degrees of freedom, where $k$ is the number of categories.  Some guidelines recommend that no more than 1/5 of the cells have expected counts less than 5, but the stricter requirement that all cells have expected counts greater than 5 is safer.

When used in this setting, the $\chi^2$ test is often called a \termsub{`goodness-of-fit' test}{goodness-of-fit test}, a term that is often misunderstood.  Small $p$-values of the test suggest evidence that a hypothesized distribution is not a good model, but non-significant $p$-values do not imply that the hypothesized distribution is the best model for the data, or even a good one. In the logic of hypothesis testing, failure to reject a null hypothesis cannot be viewed as evidence that the null hypothesis is true.

\index{data!famuss|(}

\begin{examplewrap}
\begin{nexample}{The participants in the FAMuSS study were volunteers at a university, and so did not come from a random sample of the US population.  The participants may not be representative of the general United States population. The $\chi^2$ test can be used to test the null hypothesis that the participants are racially representative of the general population. Figure~\ref{famussRacialProportions} shows the number observed by racial category in FAMuSS and the proportions of the US population in each of those categories.\footnotemark{}}

Under the null hypothesis, the sample proportions should equal the population proportions. For example, since African Americans are 0.128 of the general proportion, $(0.128)(595) = 76.16$ African Americans would be expected in the sample.  The rest of the expected counts are shown in Figure~\ref{actualExpectedRacialCountsFamuss}.

Since each expected count is greater than or equal to 5,  the $\chi^2$ distribution can be used to calculate a $p$-value for the test.

\begin{align*}
\chi^2 &= \sum_{\text{all cells}} \frac{(\text{observed} - \text{expected})^2}{\text{expected}} \\
&= \dfrac{(27-76.16)^2}{76.16} + \dfrac{(55-5.95)^2}{5.95} + \dfrac{(467-478.38)^2}{478.38} + \dfrac{(46-34.51)^2}{34.51} \\
&=440.18.
\end{align*}	

There are 3 degrees of freedom, since $k = 4$. The $\chi^2$ statistic is extremely large, and the associated tail area is smaller than 0.001. There is more than sufficient evidence to reject the null hypothesis that the sample is representative of the general population. A comparison of the observed and expected values (or the residuals) indicates that the largest discrepancy is with the over-representation of Asian participants.
\end{nexample}
\end{examplewrap}
\footnotetext{The US Census Bureau considers Hispanic as a classification separate from race, on the basis that Hispanic individuals can be any race. In order to facilitate the comparison with the FAMuSS data, participants identified as "Hispanic" have been merged with the "Other" category.}

\textD{\newpage}

\begin{figure}[h]
	\centering
	\begin{tabular}{ll ccc c ll}
		\hline
		Race	 & \hspace{2mm} & African American & Asian & Caucasian & Other & \hspace{2mm} & Total \\
		\hline
		FAMuSS &	& 27 & 55 & 467 & 46 & & 595 \\
		US Census	 & 		& 0.128 & 0.01 & 0.804 & 0.058 & & 1.00 \\
		\hline
	\end{tabular}
	\caption{Representation by race in the FAMuSS study versus the general population.}
    \label{famussRacialProportions}
\end{figure}

\begin{figure}[h]
	\centering
	\begin{tabular}{ll ccc c ll}
		\hline
		Race	 & \hspace{2mm} & African American & Asian & Caucasian & Other & \hspace{2mm} & Total \\
		\hline
		Observed &	& 27 & 55 & 467 & 46 & & 595 \\
		Expected & 	& 76.16 & 5.95 & 478.38 & 34.51 & & 595 \\
		\hline
	\end{tabular}
	\caption{Actual and expected counts in the FAMuSS data.}
    \label{actualExpectedRacialCountsFamuss}
\end{figure}	

\index{data!famuss|)}
%2004 US Census data: https://www.census.gov/population/pop-profile/dynamic/RACEHO.pdf

\index{data!Arabidopsis thaliana}

\begin{examplewrap}
\begin{nexample}{According to Mendelian genetics, alleles segregate independently; if an individual is heterozygous for a gene and has alleles $A$ and $B$, then the alleles have an equal chance of being passed to an offspring. Under this framework, if two individuals with genotype $AB$ mate, then their offspring are expected to exhibit a 1:2:1 genotypic ratio; 25\% of the offspring will be $AA$, 50\% will be $AB$, and 50\% will be $BB$. The term "segregation distortion" refers to a deviation from expected Mendelian frequencies. 
		
At a specific gene locus in the plant \textit{Arabidopsis thaliana}, researchers have observed 84 $AA$ individuals, 233 $AB$ individuals, and 134 $BB$ individuals. Is there evidence of segregation disorder at this locus? Conduct the test at $\alpha = 0.0001$ to account for multiple testing, since the original study examined approximately 250 locations across the genome.}

The Mendelian proportions are 25\%, 50\%, and 25\%. Thus, the expected counts in a group of 451 individuals are: 112.75 $AA$, 225.50 $AB$, and 112.75 $BB$. No expected count is less than 5.

\begin{align*}
\chi^2 &= \sum_{\text{all cells}} \frac{(\text{observed} - \text{expected})^2}{\text{expected}} \\
&= \dfrac{(84-112.75)^2}{112.75} + \dfrac{(233-225.50)^2}{225.50} + \dfrac{(134-112.75)^2}{112.75}\\
&=11.59.
\end{align*}
There are 2 degrees of freedom, since $k = 3$. The $p$-value is between 0.005 and 0.001, which is greater than $\alpha = 0.0001$. There is insufficient evidence to reject the null hypothesis that the offspring ratios correspond to expected Mendelian frequencies; i.e., there is not evidence of segregation distortion at this locus.
\end{nexample}
\end{examplewrap}


%___________
\section{Outcome-based sampling: case-control studies}
\label{caseControStudies}

\index{case-control studies|(}

\subsection{Introduction}

The techniques so far in this chapter have often relied on the assumption that the data were collected using random sampling from a population.  When cases come from a random sample, the sample proportion of observations with a particular outcome should accurately estimate the population proportion, given that the sample size is large enough.  When studying rare outcomes, however, moderate sized samples may contain few or none of the outcomes.  Persistent pulmonary hypertension of the newborn (PPHN) is a dangerous condition in which the blood vessels in the lungs of a newborn do not relax immediately after birth, leading to inadequate oxygenation.   The condition is rare, occurring in about 1.9 per 1,000 live births, so it is difficult to study using random sampling.  In the early 2000s, anecdotal evidence began to accumulate that the risk of the condition might be increased if the mother of the newborn had been taking a particular medication for depression, a selective serotonin reuptake inhibitor (SSRI) during the third trimester of pregnancy or even as early as during week 20 of the pregnancy.  

One design for studying the issue would enroll two cohorts of women, one in which women were taking SSRIs for depression and one in which they were not. However, if the chance of PPHN was 1.9/1,000 in newborns of a control cohort of 1,000 women, then the probability of observing no cases of PPHN is about 0.15. If the probability of PPHN is elevated among infants born to women taking SSRIs, such as to 3.0/1,000, the chance of observing no cases among 1,000 women is approximately 0.05. Precise measures of the probability of PPHN occurring would require very large cohorts.

\index{sample!outcome-dependent}

An alternative design for studies like this reverses the sampling scheme so that the two cohorts are determined by outcome, rather than exposure; a cohort with the condition and a cohort without the condition are sampled, then exposure to a possible cause is recorded. To apply this design for studying PPHN, a registry of live births could be used to sort births by presence or absence of PPHN. The number in each group in which the mother had been taking SSRIs could then be recorded (based on medical records). Such a design would have the advantage of sufficient numbers of cases with and without PPHN, but it has other limitations which will be discussed later in this section. Traditionally, these studies have been called case-control studies because of the original sampling of individuals with and without a condition.  More generally, it is an example of outcome-dependent sampling.


\textD{\newpage}


\subsection{$\pmb{\chi^2}$ tests of association in case-control studies}
\label{caseControlTests}

\index{case-control studies!tests for association}
\index{data!persistent pulmonary hypertension in newborns (PPHN)|(}

In 2006, Chambers, et. al reported a case-control study examining the association of SSRI use and persistent pulmonary hypertension in newborns.\footnote{N Engl J Med 2006;354:579-87.}  The study team enrolled 337 women whose infants suffered from PPHN and 836 women with similar characteristics but whose infants did not have PPHN.  Among the women whose infants had PPHN, 14 had taken an SSRI after week 20 of the pregnancy.  In the cohort of women whose infants did not have PPHN, 6 had been taking the medication after week 20. In the subset of women who had been taking an SSRI, the infants are considered `exposed' to the medication. The data from the study are summarized in Figure~\ref{ssriPPHNObserved}.

\begin{figure}[h]
	\centering
	\begin{tabular}{ll rrr r}
		\hline
		PPHN present	 & \hspace{2mm} & Yes & No & \hspace{2mm} & Total \\
		\hline
		SSRI exposed &	& 14 & 6 &  & 20  \\
		SSRI unexposed & & 323 & 830 &  & 1153  \\
        Total & & 337 & 836 & & 1173 \\
		\hline
	\end{tabular}
	\caption{SSRI exposure vs observed number of PPHN cases in newborns.}
    \label{ssriPPHNObserved}
\end{figure}	

The sample of women participating in the study are clearly not a random sample drawn from women who had recently given birth; they were identified according to the disease status of their infants.  In this sample, the proportion of newborns with PPHN (337/1173 = 28.7\%) is much higher than the disease prevalence in the general population.  

Even so, the concept of independence between rows and columns under a null hypothesis of no association still holds. If SSRI use had no effect on the occurrence of PPHN, then the proportions of mothers taking SSRIs among the PPHN and non-PPHN infants should be about the same. In other words, the null hypothesis of equal SSRI use among mothers with/without PPHN affected infants is the hypothesis of no association between SSRI use and PPHN. The test of independence can be conducted using the approach introduced earlier in the chapter.

The expected counts shown in Figure~\ref{ssriPPHNExpected} suggest that the $p$-value from a $\chi^2$ test may not be accurate; under the null hypothesis, the expected number of PPHN cases in the SSRI exposed group is less than 10.

 \begin{figure}[h]
	\centering
	\begin{tabular}{ll rrr r}
		\hline
		PPHN present  & \hspace{2mm} & Yes & No & \hspace{2mm} & Total \\
		\hline
		SSRI exposed &	& 5.80 & 14.20 &  & 20  \\
		SSRI unexposed & & 331.20 & 811.80 &  & 1153  \\
        Total & & 337 & 836 & & 1173 \\
		\hline
	\end{tabular}
    \caption{SSRI exposure vs expected number of PPHN cases in newborn.}
    \label{ssriPPHNExpected}
\end{figure}	

The $p$-value from Fisher's exact test is $< 0.001$ ($0.00014$, to be precise), so the evidence is strong that SSRI exposure and PPHN are associated. Fisher's exact test is often used in studies of rare conditions or exposures since one or more expected cell counts are typically less than 10.

\index{data!persistent pulmonary hypertension in newborns (PPHN)|)}


\textD{\newpage}


\subsection{Estimates of association in case-control studies}
\label{caseControlStudiesEstimates}

\index{odds ratio!case-control studies}

For data in a $2 \times 2$ table, correct point estimates of association depend on the mechanism used to gather the data.  In the example of a clinical trial of nevirapine versus lopinarvir discussed in Section~\ref{twoWayTablesExpectedCounts}, the population proportion of children who would experience virologic failure after treatment with one of the drugs can be estimated by the observed proportion of virologic failures while on that drug. For nevirapine, the proportion of children with virologic failure is 60/147 = 0.41, while for lopinarvir the proportion is 27/140 = 0.19.  The difference in outcome between the two groups can be summarized by the difference in these proportions. The proportion experiencing virologic failure when treated with nevirapine was 0.12 larger in nevirapine (0.41 - 0.29), so if the two drugs were to be used in a large population, approximately 12\% more children treated with nevirapine would experience virologic failure as compared to lopinarvir. The confidence intervals discussed in Section~\ref{confidenceIntervalsDifferenceProportions} can be used to express the uncertainty in this estimate.

Since the proportion of virologic failures can be estimated from the trial data, the relative risk of virologic failure can also be used to estimate the association between treatment and virologic failure. Relative risk is the ratio of two proportions, and was introduced in Section~\ref{TwoWayTablesRelativeRisk}.  The relative risk of virologic failure with nevirapine versus lopinarvir is $0.41/0.19 = 2.16$.  Children treated with nevirapine are estimated to be more than twice as likely to experience virologic failure.  

Statistically, the population parameter for the relative risk in the study of HIV$^+$ is a ratio of conditional probabilities:

\begin{align*}
  \frac{P(\text{virologic failure}| \text{treatment with nevirapine})}
  {P(\text{virologic failure}|\text{treatment with lopinarvir})}.
\end{align*}

In a study like the PPHN case-control study, the natural population parameter of interest would be the relative risk of PPHN for infants exposed to an SSRI during after week 20 of gestation compared to those who were not exposed. However, in the design of this study, participating mothers were sampled and grouped according to whether their infants did or did not suffer from PPHN, rather than assigned to either SSRI exposure or non-exposure. Relative risk of PPHN from exposure to SSRI cannot be estimated from the data because it is not possible to estimate $P(\text{PPHN} | \text{SSRI exposure})$ and $P(\text{PPHN} | \text{no SSRI exposure})$. In case-control studies, association is estimated using \term{odds} and \termsub{odds ratios}{odds ratio} rather than relative risk.

The \term{odds} of SSRI exposure among the cases are given by the fraction
\[
  \text{odds$_\text{cases}$} = \frac{P(\text{SSRI exposure} | \text{PPHN})}
  {P(\text{no SSRI exposure} | \text{PPHN})} = \frac{14/337}{323/337} = \frac{14}{323}.
\]

The odds of SSRI exposure among the controls are given by the fraction
\[
  \text{odds$_\text{controls}$} = \frac{P(\text{SSRI exposure} | \text{no PPHN})}
  {P(\text{no SSRI exposure} | \text{no PPHN})} = \frac{6/836}{830/836} = \frac{6}{830}.
\]

The ratio of the odds, the odds ratio, compares the odds of exposure among the cases to the odds of exposure among the controls: 

\[OR_{\text{exposure, cases vs. controls}} = \frac{\text{odds$_\text{cases}$}}{\text{odds$_\text{controls}$}} = \frac{14/323}{6/830} = \frac{(14)(830)}{(323)(6)} =  6.00. \]

\textD{\newpage}

A population odds ratio of, for example, 1.5, implies that the odds of exposure in cases are 50\% larger than the odds of exposure in controls. For this study, the odds ratio of 6.00 implies that the odds of SSRI exposure in infants with PPHN are 6 times as large as the odds of exposure in infants without PPHN. Epidemiologists describe this odds ratio as the odds of exposure given presence of PPHN compared to the odds of exposure given absence of PPHN. An OR greater than 1 suggests that the exposure may be a risk factor for the disease or condition under study.  Epidemiologists also use the term \term{relative odds} as a synonym for odds ratio.

Surprisingly, the odds ratio of exposure comparing cases to controls is equivalent to the odds ratio of disease comparing exposed to unexposed.\footnote{This result can be shown through Bayes' rule.} With a specific example, it is easy to see how the fraction for the odds ratios are numerically equivalent:

\[OR_{\text{disease, exposed versus unexposed}} = \frac{\text{odds$_\text{exposed}$}}{\text{odds$_\text{unexposed}$}} = \frac{14/6}{323/830} = \frac{(14)(830)}{(6)(323)} = 6.00. \]

Despite the apparently restrictive nature of the case-control sampling design, the odds ratio of interest, the odds ratio for disease given exposure, can be estimated from case-control data.

Epidemiologists rely on one additional result, called the rare disease assumption. When a disease is rare, the odds ratio for the disease given exposure is approximately equal to the relative risk of the disease given exposure.  These identities are the reason case-control studies are widely used in settings in which a disease is rare: it allows for the relative risk of disease given exposure to be estimated, even if the study design is based on sampling cases and controls then measuring exposure.

In a general $2 \times 2$ table of exposure versus disease status (Figure~\ref{generalTwoByTwoTable}) the odds ratio for disease given exposure status is the $ad/bc$.
 \begin{figure}[h]
	\centering
	\begin{tabular}{ll rrr r}
		\hline
		Disease Status  & \hspace{2mm} & Present & Absent & \hspace{2mm} & Total \\
		\hline
		Exposed &	& $a$ & $b$ &  & $a + b$  \\
		Unexposed & & $c$ & $d$ &  & $c + d $  \\
        Total & & $a + c$ & $b + d$ & & $n$ \\
		\hline
	\end{tabular}
    \caption{Exposure vs Disease Status.}
    \label{generalTwoByTwoTable}
\end{figure}	

In the PPHN case-control data, the odds ratio for PPHN given SSRI exposure status is $(14)(830)/(6)(323) = 6.00$.  Because PPHN is a rare condition, the risk of PPHN among infants exposed to an SSRI is estimated to be approximately 6 times that of the risk among unexposed infants.  Infants exposed to an SSRI are 600\% more likely to suffer from PPHN.  

It can be shown that the $p$-value used in a test of no association (between exposure and disease) is also the $p$-value for a test of the null hypothesis that the odds ratio is 1.

\index{case-control studies|)}


\section{Notes}
\label{infForPropNotes}

Two-way tables are often used to summarize data from medical research studies, and entire texts have been written about methods of analysis for these tables.  This chapter covers only the most basic of those methods. 

Until recently, Fisher's exact test could only be calculated for $2 \times 2$ tables with small cell counts. Research has produced faster algorithms for enumerating tables and calculating $p$-values, and the computational power of recent desktop and laptop computers now make it possible to calculate the Fisher test on nearly any $2 \times 2$ table.  There are also versions of the test that can be calculated on tables with more than 2 rows and/or columns.  The practical result for data analysts is that the sample size condition for the validity of the $\chi^2$ test can be made more restrictive.  This chapter recommends using the $\chi^2$ test only when cell counts in a $2 \times 2$ table are greater than 10; some approaches recommend cell counts larger than 10.

For many years, introductory textbooks recommended using a modified version of the $\chi^2$ test, called the Fisher-Yates test, which adjusted the value of the statistic in small sample sizes to increase the accuracy of the $\chi^2$ sampling distribution in calculating $p$-values.  The Fisher-Yates version of the test is no longer used as often because of the widespread availability of the Fisher test.

The Fisher test is not without controversy, at least in the theoretical literature.  Conditioning on the row and column totals allows the calculation of a $p$-value from the hypergeometric distribution, but in principle restricts inference to the set of tables with the same row and column values.  In practice, this is less serious than it may seem. For tables of moderate size, the $p$-values from the $\chi^2$ and Fisher tests are nearly identical and for tables with small counts, the Fisher test guarantees that the Type I error will be no larger than the specified value of $\alpha$.   In small sample sizes, some statisticians argue that the Fisher-Yates correction is preferable to the Fisher test because of the discrete nature of the hypergeometric distribution.  In small tables, for example, an observed $p$-value of 0.04 may be the largest value that is less than 0.05, such that the Type I error of the test in that situation is 0.04, not 0.05.

Section~\ref{caseControlStudiesEstimates} does not show the derivation that the odds ratio estimated from a case-control is the same as that from a cohort study. It is long and algebraically more complex than other derivations shown in the text, but it is a direct application of Bayes' rule, applied to each term in the fraction that defines population odds ratio.

The two labs for this chapter examine methods of inference for the success probability in binomial data then generalizes inference for binomial proportions to two-way contingency tables.  Lab 2 also discusses measures of association in two-by-two tables.  The datasets in the labs are similar to datasets that arise frequently in medical statistics.  Lab 1 assesses the evidence for a treatment effect in a single uncontrolled trial of a new drug for melanoma and whether outcomes in stage 1 lung cancer are different among patients treated at Dana-Farber Cancer Institute compared to population based statistics. In Lab 2, students analyze a dataset from a published clinical trial examining the benefit of using a more expensive but potentially more effective drug to treat HIV-positive infants.

